\documentclass[]{article}
\usepackage[T1]{fontenc}
\usepackage{lmodern}
\usepackage{amssymb,amsmath}
\usepackage{ifxetex,ifluatex}
\usepackage{fixltx2e} % provides \textsubscript
% use upquote if available, for straight quotes in verbatim environments
\IfFileExists{upquote.sty}{\usepackage{upquote}}{}
\ifnum 0\ifxetex 1\fi\ifluatex 1\fi=0 % if pdftex
  \usepackage[utf8]{inputenc}
\else % if luatex or xelatex
  \ifxetex
    \usepackage{mathspec}
    \usepackage{xltxtra,xunicode}
  \else
    \usepackage{fontspec}
  \fi
  \defaultfontfeatures{Mapping=tex-text,Scale=MatchLowercase}
  \newcommand{\euro}{€}
\fi
% use microtype if available
\IfFileExists{microtype.sty}{\usepackage{microtype}}{}
\usepackage{longtable,booktabs}
\ifxetex
  \usepackage[setpagesize=false, % page size defined by xetex
              unicode=false, % unicode breaks when used with xetex
              xetex]{hyperref}
\else
  \usepackage[unicode=true]{hyperref}
\fi
\hypersetup{breaklinks=true,
            bookmarks=true,
            pdfauthor={Syllabus},
            pdftitle={Workplace and Apprenticeship 10 Period 5},
            colorlinks=true,
            citecolor=blue,
            urlcolor=blue,
            linkcolor=magenta,
            pdfborder={0 0 0}}
\urlstyle{same}  % don't use monospace font for urls
\setlength{\parindent}{0pt}
\setlength{\parskip}{6pt plus 2pt minus 1pt}
\setlength{\emergencystretch}{3em}  % prevent overfull lines
\setcounter{secnumdepth}{0}

\title{Workplace and Apprenticeship 10 Period 5}
\author{Syllabus}
\date{Mr.~Guenther}

\begin{document}
\maketitle

\section{Goals}\label{goals}

\begin{itemize}
\itemsep1pt\parskip0pt\parsep0pt
\item
  Learn to love math and use it in real life
\item
  Confidently pass the course
\end{itemize}

\section{Outline}\label{outline}

\begin{longtable}[c]{@{}ll@{}}
\toprule\addlinespace
Day & Lesson
\\\addlinespace
\midrule\endhead
Week 1 & Equalities
\\\addlinespace
Week 2 & Spatial Reasoning
\\\addlinespace
Week 3 & Measuring Systems
\\\addlinespace
\bottomrule
\end{longtable}

\section{Grades}\label{grades}

\begin{longtable}[c]{@{}ll@{}}
\toprule\addlinespace
Type & Weight
\\\addlinespace
\midrule\endhead
Quizs & 70\%
\\\addlinespace
Projects & 30\%
\\\addlinespace
\bottomrule
\end{longtable}

While attendance and interaction in class are not marked, I can
guarantee you that showing up to class and being attentive will result
in a higher grade on your mark, because you will be learning.

\subsection{Quiz}\label{quiz}

Every Friday will be a quiz, summarizing what has been learned
throughout the week. The quiz will be a maximum of 20 minutes each week,
and will happen at the later part of the class.

The quizzes will follow the curriculum for assessment.

Each question on the quiz will be worth 3 marks.

\subsubsection{Quiz Grade}\label{quiz-grade}

\begin{longtable}[c]{@{}ll@{}}
\toprule\addlinespace
Mark & Translation
\\\addlinespace
\midrule\endhead
0 & Nothing
\\\addlinespace
1 & + Right Equations
\\\addlinespace
2 & + Right Steps
\\\addlinespace
3 & + Right Answer
\\\addlinespace
\bottomrule
\end{longtable}

Questions on the quizzes will be sorted according to content type.
Similar questions will be asked multiple times in order for the student
to show achievement. The best grade will be taken.

\subsubsection{Example}\label{example}

On my first quiz I got 1 out of 3 on the Pythagorean theorem, but did
well on the other questions. The next quiz was the opposite, I did well
on the Pythagorean theorem, but didn't do so well on the others. In this
case, the highest grade will be taken from each section of the 2
quizzes, to give the student the highest possible grade.

\subsubsection{Preparing for Quizzes}\label{preparing-for-quizzes}

As you may have noticed by now, there are not assignments in this class.
The only thing you will be marked on is your projects and your quizzes.
However, if your grade in this class will greatly increased if you
practice with assignments. At times I will give out questions to
practice with, and then review them on the board. You are not obligated
to do this, but if you refuse to work on them, your grades in this class
will suffer.

Additionally, anyone who's grade is a \textbf{D} will have these
assignments be mandatory. Failure to turn in a completed assignment will
result in a -1\% off your final grade. Once you manage to raise your
grade above a \textbf{D} all your penalties will be cleared.

If at the end of the semester I decide to review this policy in your
favour, I reserve the right to do so.

\subsection{Projects}\label{projects}

Projects are meant to be real life applications for using math. Each
project will be in the form of a question

It is then your task to find the answer. Initially I will give you a lot
of direction, but as the semester continues you will need to provide
your own direction in order to complete these projects.

\subsubsection{Example}\label{example-1}

How many baseballs would fit in a baseball diamond?

Judging from your current income and expenses, how long before you reach
1 million dollars?

\subsubsection{Grade}\label{grade}

\begin{longtable}[c]{@{}ll@{}}
\toprule\addlinespace
Mark & Translation
\\\addlinespace
\midrule\endhead
0 & Nothing is done
\\\addlinespace
1 & + Attempted
\\\addlinespace
3 & + Correct use of equations
\\\addlinespace
4 & + Shows work
\\\addlinespace
5 & + Correct Solution with no errors
\\\addlinespace
6 & + Adds presentation
\\\addlinespace
7 & + Perfection (I can't find anything to fix)
\\\addlinespace
\bottomrule
\end{longtable}

\subsubsection{Due Dates}\label{due-dates}

Projects will be due on the day I set for them. There is a grade of 0
given if the project is not turned in on this day.

The reason being when you get into the workplace , there are deadlines,
and you could lose your job if you fail to meet those guides.

That said, my purpose here is to prepare you for this task. I will
provide a tentative deadline. Some companies actually do this to make
sure the project is turned in on time. If you hand in your work ahead of
the final deadline, at your request I can look over your project and
recommend any changes before I actually grade it.

I will not do this if you turn in the assignment on the due date.

\section{Closing Remarks}\label{closing-remarks}

it is my commitment to you to help you as you discover more about math.
I cannot help you if you are not engaged in class. I need your
cooperation to make this work. I need your committment to work hard on
your projects and quizzes.

\subsection{Food}\label{food}

Food is an essential part of your intellectual and physical growth, both
of which are necessary for your spiritual growth. As such I have a
recommended diet

\begin{itemize}
\itemsep1pt\parskip0pt\parsep0pt
\item
  Fish (Omega 3 for enhancing memory)
\item
  Eggs (vitamins A \& B are good for memory)
\item
  Caffeine before class for building new memoires
\item
  Cherries for falling asleep
\item
  Bananas to calm and relax muscles
\item
  Almonds to promote sleep and muscle relaxation
\item
  Dark chocolate for stress relief
\item
  Avocados for the immune system
\item
  Garlic and Onions for the win (maybe ask your parents about this one
  :)
\end{itemize}

\subsection{Sleep}\label{sleep}

I recommend 8 hours a night. You need your brain functioning at full
capacity as you study math.

\subsection{Relaxation}\label{relaxation}

You brain needs to work hard at school. That is why I recommend that you
take at least 1 hour a night doing something that relaxes you.

Some things that relax me are as such: * video games * reading * TV show
* discovering new technology * taking pictures * building with lego

some things that I DON'T find relaxing: * Surfing facebook * doing
homework

\subsection{End}\label{end}

This can be a great semester, but it will not be easy all the time. I
need your cooperation to make this a fruitful semester.

Sincerly,

David Guenther

\end{document}
